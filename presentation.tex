% Slides for ApacheCon US 2015
%
% Phil Steitz <phil@steitz.com>
% August 3, 2016

\documentclass[14pt,mathserif]{beamer}
\usetheme{Boadilla}

\setbeamertemplate{footline}
{
  \leavevmode%
  \hbox{%
    \begin{beamercolorbox}[wd=.333333\paperwidth,ht=2.25ex,dp=1ex,center]{author in head/foot}%
    \usebeamerfont{author in head/foot}\insertshortinstitute
    \end{beamercolorbox}%
    \begin{beamercolorbox}[wd=.333333\paperwidth,ht=2.25ex,dp=1ex,center]{title in head/foot}%
      \usebeamerfont{title in head/foot}\insertshorttitle
    \end{beamercolorbox}%
    \begin{beamercolorbox}[wd=.333333\paperwidth,ht=2.25ex,dp=1ex,right]{date in head/foot}%
      \usebeamerfont{date in head/foot}\insertshortdate{}\hspace*{2em}
      \insertframenumber{} / \inserttotalframenumber\hspace*{2ex}
    \end{beamercolorbox}}%
  \vskip0pt%
}

\usepackage[T1]{fontenc}
\usepackage{lmodern}
\usepackage{color}
\usepackage{minted}

%---- Copy the frame title color --------------------------%
\newcommand{\fblue}[1]{\usebeamercolor[fg]{frametitle}{#1}}

\newcommand{\newauthor}[2]{
  \parbox{0.26\textwidth}{
    \texorpdfstring
      {
        \centering
        #1 \\
        {\scriptsize{\urlstyle{same}\url{#2}\urlstyle{tt}}}
      }
      {#1}
  }
}

\hypersetup{colorlinks,linkcolor=cyan,urlcolor=blue}

%----------- Section title slides -----------------------------------%
\AtBeginSection[]{
  \begin{frame}
  \vfill
  \centering
  \begin{beamercolorbox}[sep=8pt,center,shadow=true,rounded=true]{title}
    \usebeamerfont{title}\secname\par%
  \end{beamercolorbox}
  \vfill
  \end{frame}
}

%gets rid of navigation symbols
\setbeamertemplate{navigation symbols}{}

\title{Java as a Platform for Statistical Computing}
\author{
  \newauthor{Phil Steitz}{phil@steitz.com}
}
\institute[Joint Statistical Meetings]{Joint Statistical Meetings 2016}
\date{August 3, 2016}
\begin{document}

%----------- titlepage ----------------------------------------------%
{
\begin{frame} %[plain]
  \titlepage
\end{frame}
}

%----------- slide --------------------------------------------------%
\begin{frame}
  \frametitle{Agenda}
\begin{itemize}
  \item Historical challenges
  \item Contemporary improvements: language, VM, libraries
  \item Distributed computing frameworks
  \item Example: BLB implemented using Apache Spark and Hipparchus Statistics
\end{itemize}

\end{frame}

%--------------------------------------------------------------------%
\section[History]{Historical Challenges}

%----------- slide --------------------------------------------------%
\begin{frame}
  \frametitle{Historical Criticisms of Java for Scientific Computing}

\begin{enumerate}
  \item Java is slow
  \item Java numerics are poorly defined and implemented
  \item Robust, fully tested implementations of core algorithms is lacking
\end{enumerate}

\end{frame}

%----------- slide --------------------------------------------------%
\begin{frame}
  \frametitle{Java Performance}
Early 1.1, 1.2 performance. Analysis and benchmarks from, e.g. http://dl.acm.org/citation.cfmdoid=304065.304092
\\
\\
Brief overview of core performance limiters and progression of JDKs, as in
https://en.wikipedia.org/wiki/Java\_performance
\end{frame}

%----------- slide --------------------------------------------------%
\begin{frame}
  \frametitle{Java Numerics}
Follow progression in
Kahan, W.; Joseph D. Darcy (1998-03-01). "How Java's Floating-Point Hurts Everyone Everywhere"

\end{frame}

%----------- slide --------------------------------------------------%
\begin{frame}
  \frametitle{Java Libraries}
Quick overview of java.lang.Math and early efforts 
\begin{enumerate}
\item http://math.nist.gov/javanumerics/
\item https://dst.lbl.gov/ACSSoftware/colt/
\item http://math.nist.gov/javanumerics/jama/
\end{enumerate}
Examples showing big gap between what has been available in Java and Fortran, C, Python, R
\end{frame}

%--------------------------------------------------------------------%
\section[History]{Contemporary Improvements}

%----------- slide --------------------------------------------------%
\begin{frame}
  \frametitle{Performance}
  Summary of core performance improvements
  \begin{enumerate}
  \item JIT
  \item Threading
  \item IO
  \item Optimizer
  \end{enumerate}
\end{frame}

%----------- slide --------------------------------------------------%
\begin{frame}
  \frametitle{Numerics and Libraries}
  Language improvements and "standard" workarounds for floating point anomalies baked 
  into libraries such as Apache Commons Math, Mahout, Sanselan.  
  \\
  \\
  Review of key libraries. Multiple slides on current generation java math/stat libraries.  Careful review of Apache Commons Math, Apache Mahout / Sanselan and Hipparchus (new).
\end{frame}

%--------------------------------------------------------------------%
\section[Distributed Computing]{Distributed Computing Frameworks}

%----------- slide --------------------------------------------------%
\begin{frame}
  \frametitle{BigData means Distributed Computing}
Three slides showing architectures of Apache Hadoop, Apache Spark and Apache Flink.
\\
\\
One slide summarizing core architecture requirements: data distribution, process coordination, inter-process communication and Java language features that support these.
\\
\\
One slide summarizing Java middleware / supporting OSS infrastructure underneath key frameworks

\end{frame}

%--------------------------------------------------------------------%
\section[Big Data Example]{Example: Bag of Little Bootstraps}

%----------- slide --------------------------------------------------%
\begin{frame}
  \frametitle{Problem: Bootstrapping Very Large Datasets}
Present a simple estimation problem: confidence intervals for a quantile estimate.  
\\
\\
Motivate this as SLA determination problem. Huge stream of API response times.
\end{frame}

%----------- slide --------------------------------------------------%
\begin{frame}
\frametitle{Simple solution: pool all data and bootstrap}
Present full dataset bootstrap solution algorithm.
\end{frame}

%----------- slide --------------------------------------------------%
\begin{frame}
\frametitle{Problem: intractable for very large datasets}
Do arithmetic on memory, data movement, resampling cost.
\end{frame}

%----------- slide --------------------------------------------------%
\begin{frame}
  \frametitle{Solution: Bag of Little Bootstraps}
  \begin{itemize}
  \item Basic idea: partition the large dataset into subsamples and average bootstrap estimates
        computed over subsamples
  \item Requires less space, data movement and computing 
  \item \textcolor{blue}
      {\href{https://arxiv.org/pdf/1112.5016.pdf}{A Scalable Bootstrap for Massive Data}}*
      provides theoretical justification and empirical analysis of space and compute saves
  \end{itemize}
  \begin{small}
  *Kleiner, Talwalkar, Sarkar, Jordan, Journal Of The Royal Statistical Society 76(4), 2011
  \end{small}
\end{frame} 

%----------- slide --------------------------------------------------%
\begin{frame}
  \frametitle{Application to current problem}
  \begin{enumerate}
  \item Shard large stream of input data into random subsamples
  \item Compute bootstrap confidence intervals for each subsample
  \item Average the estimated upper and lower bounds across the subsamples
  \end{enumerate}
 \end{frame}
  
%----------- slide --------------------------------------------------%
\begin{frame}
  \frametitle{Logical Components Required for Solution}
  \begin{enumerate}
  \item Supervisor to split out subsamples, distribute to workers and aggregate results
  \item Data management 
  \item Quantile estimation
  \item Inter-process communication stack
  \item Process coordination
  \item Configuration management
  \end{enumerate}
\end{frame}
  
%----------- slide --------------------------------------------------%
\begin{frame}
  \frametitle{Implementation Leveraging Java Frameworks}
Diagram showing solution architecture, based on Apache Spark and Zookeeper with Hipparchus (extension of Apache Commons Math) for estimation.
\\
\\
Follow with samples of implementation code
\end{frame}

%----------- slide --------------------------------------------------%
\begin{frame}
  \frametitle{Conclusion}
\begin{enumerate}
\item Summarize discussion of limitations and how contemporary JDKs and supporting libraries have addressed them.
\item Summarize big data example, pointing out ease of implementation in Java.  
\item Also point out that hybrid solutions can be easy and interoperability is becoming easy.  Show R-Java bindings as example and execution of R jobs inside Spark.
\end{enumerate}
\end{frame}

\end{document}